\chapter{Entre o Dogma e o Documento}
\label{cap:dogmadocumento}

\section{A fé redigida}

A história da Igreja prova que, desde os tempos apostólicos, a necessidade de registrar a fé por escrito sempre foi uma resposta à confusão doutrinária. A \gls{cfw}, por exemplo, nasce em um contexto de debate, instrução e ordenação teológica — não para limitar o Espírito, mas para proteger a sã doutrina da criatividade excessiva dos que se dizem guiados por Ele.

\begin{citacao}
``Deus, de acordo com o seu beneplácito, revelou-se gradualmente e fez com que a sua vontade fosse inteiramente escrita para que o conhecimento da salvação fosse preservado e propagado.'' (\gls{cfw}, cap. I, art. I)
\end{citacao}

É nessa tradição que se insere o presente trabalho: não como inovação, mas como continuidade do esforço cristão por pensar, escrever e viver segundo a verdade revelada — ainda que com recuo de parágrafo e sumário automático.

\subsection{O desafio da coerência no século XXI}

Em tempos de pós-verdade, a coerência doutrinária virou uma espécie de resistência teológica. Segundo \citeonline{stottContracultura}, o cristianismo autêntico caminha na contramão da cultura, o que inclui resistir à tentação de ajustar os princípios bíblicos ao gosto da audiência ou ao cronograma da formatação.

\begin{citacao}
"Vivemos em um mundo que cada vez mais rejeita absolutos. Mas a fé cristã — a verdadeira — é uma contracultura com convicções firmes." \cite{stottContracultura}
\end{citacao}

Assim, escrever uma monografia que respeite a \gls{cfw}, dialogue com a tradição reformada e ainda consiga passar no filtro da banca é, por si só, um exercício de fidelidade.

\subsubsection{Entre os artigos da fé e os artigos da ABNT}

A tensão entre a estrutura acadêmica e o conteúdo teológico não é recente, mas se acentua nos corredores dos seminários, onde se espera que o seminarista demonstre, com igual domínio, a doutrina da expiação e o uso correto de `\texttt{\textbackslash cite}`. 

É necessário lembrar que \textbf{a fé não se resume a notas de rodapé}, mas que \underline{quando bem aplicadas}, elas ajudam a evitar heresias interpretativas e acusações de plágio \footnote{ambas gravíssimas no contexto presbiteriano}.

\section{Síntese do capítulo}

Este capítulo mostrou que a tradição reformada valoriza o conteúdo e a forma — não apenas no culto, mas também na produção escrita. A \gls{cfw}, os autores clássicos e os contemporâneos, como \citeonline{stottContracultura}, indicam que viver e escrever a fé são atos igualmente sérios. Por isso, que esta monografia sirva como um testemunho: da graça que sustenta, da verdade que permanece, e da banca que — por misericórdia — aprova.

