\chapter{A Importância do Tema}
\label{cap:importancia}

\subsection{A relação entre fé e formatação}

A tradição reformada sempre valorizou a ordem e a decência no culto (\gls{cfw}), mas nem sempre se lembram que tais princípios se aplicam também à estrutura do \textit{Trabalho de Conclusão de Curso}. Escrever sob a orientação do \imprimirorientador é um ato de submissão voluntária e uma expressão prática da doutrina da \textit{perseverança dos santos}.

Como lembra o apóstolo Paulo, \textit{``tudo, porém, seja feito com decência e ordem''} (\gls{1co} 14.40, \gls{naa}). Esse princípio, ainda que dirigido à liturgia da igreja, pode muito bem iluminar também o labor acadêmico, que exige do seminarista zelo tanto pelo conteúdo quanto pela forma.

\begin{citacao}
    A boa teologia não se mede apenas pelo conteúdo, mas pela clareza com que é apresentada, especialmente quando há prazos, banca e normas contraditórias a serem consideradas.
\end{citacao}

Segundo \citeonline[cap.~4, pp.~77-80]{calvinoInstitutas}, Deus governa todas as coisas com sabedoria perfeita.

\subsubsection{A figura do presbítero-leigo}

O papel do \textbf{presbitério} na formação teológica é inegável. Contudo, a atuação de presbíteros-leigos que se tornam ``especialistas em críticas exegéticas'' após três vídeos no YouTube é um fenômeno digno de investigação. Conforme aponta \cite{bavinckDogmatica}, a teologia sistemática exige método, fontes e humildade — especialmente esta última, quando se decide emitir pareceres públicos sobre assuntos que se conhece apenas por intuição denominacional.\footnote{Não basta ter zelo. É preciso ter base. E nem toda base teológica vem embalada em vídeos curtos e devocionais polêmicos.}
