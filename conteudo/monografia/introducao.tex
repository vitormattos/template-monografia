\newpage
\newcommand{\tituloIntroducao}{Introdução}

\chapter*{\tituloIntroducao} % O asterístico suprime a numeração da introdução no sumário
\markboth{\tituloIntroducao}{\tituloIntroducao} % pelo fato da numeração da introdução ter sido suprimida, é preciso isto para que o cabeçalho da página da introdução fique com o título correto
\addcontentsline{toc}{section}{\MakeUppercase{\tituloIntroducao}}

Durante uma conversa de corredor — ambiente onde, curiosamente, surgem os temas mais polêmicos da vida teológica — alguém indagou se seria possível desenvolver uma monografia que fosse ao mesmo tempo fiel às Escrituras, à tradição reformada e às exigências da ABNT. Não houve resposta clara, mas o desafio foi lançado.

O tema escolhido, embora não tenha nascido de revelação nem de voto presbiterial, floresceu no solo fértil da obrigação curricular. Ele trata de um dilema que muitos evitam, poucos compreendem e quase todos criticam: a relação entre a formação teológica, o campo de trabalho e as pressões, incluindo aqui orientadores, professores e presbitérios, sempre prontos a emitir pareceres doutrinários baseados em fortes convicções pessoais e leituras da \gls{cfw}.

A hipótese de trabalho, ainda em oração e revisão, parte do princípio de que é possível entregar esta monografia dentro do prazo. Parte-se também da crença reformada de que Deus é soberano sobre todas as coisas, inclusive sobre o julgamento da banca, mas que isso não isenta o seminarista da responsabilidade de obedecer às normas de formatação.

A relevância do tema está diretamente ligada à sobrevivência acadêmica do autor e à esperança de que esta obra sirva como modelo (ou aviso) para futuras gerações e para apresentar ao presbitério compostos de amados reverendíssimos. A análise aqui empreendida pretende lançar luz sobre aspectos muitas vezes negligenciados da caminhada teológica, como a influência das decisões do presbitério na vida devocional e o impacto das reuniões da banca na doutrina da perseverança dos santos.

O objetivo geral deste trabalho é claro: entregar, ser aprovado e, com um pouco de sorte, nunca mais ser citado. Especificamente, pretende-se demonstrar que o esforço de sistematização teológica em ambiente acadêmico pode conviver com uma crítica bem-humorada à estrutura que o sustenta.

A presente monografia está organizada em três capítulos, cada um mais ousado que o anterior. O primeiro trata da gênese do tema, suas motivações e conflitos eclesiológicos internos. O segundo aprofunda a análise da tensão entre teoria e prática ministerial. O terceiro, por fim, propõe um caminho de reconciliação entre a rigidez institucional e a leveza da graça, mesmo que para isso seja necessário escrever vinte páginas sobre um assunto que poderia ser resolvido com um bom sermão de domingo.

A metodologia é eminentemente bibliográfica por questões óbvias, pois o seminarista não é um ser pensante e apenas deve dialogar com respeito com os autores, conectando suas ideias, apoiando-se em livros, artigos, atas de concílio, manuais de redação acadêmica e uma quantidade respeitável de anotações em margens de apostilas. O uso de fontes primárias e secundárias, com aspas ou sem, dependerá do humor do orientador no dia da leitura.

