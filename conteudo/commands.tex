\instituicao{Instituto Superior de Altíssima Reverência e Santidade Aplicada}
\titulo{Monografia teológica reformada}
\newcommand{\subtema}{Modelo estruturado segundo as normas da ABNT e os princípios da Igreja Presbiteriana do Brasil}
\autor{Eclesionildo Reverendíssimo Green dos Calvinuston}
\data{2025}
\local{Vila Predestinada do Norte}
\newcommand{\dataAprovacao}{\rule{1cm}{0.8pt} de \rule{3cm}{0.8pt} de \imprimirdata}
\newcommand{\palavrasChave}{unção acadêmica. homilética coach. conferência do ego. vaidade teológica. culto da estética.}
\newcommand{\keywords}{church, Christian ethics, technology, privacy, data security.}
\orientador{Rev. Agostinélson da Confissão Imutável}
\tipotrabalho{Monografia}
\preambulo{Trabalho acadêmico elaborado para a \textbf{\imprimirinstituicao}, como parte do rito de passagem reformado e obrigatório, sob a gloriosa orientação do inesquecível \imprimirorientador, também conhecido como \textit{``aquele que corrige com amor e a caneta vermelha''}.}
\newcommand{\notaApresentacaoProjeto}{Pré-projeto monográfico apresentado à \textbf{\imprimirinstituicao}, porque, segundo o regulamento, o chamado não basta, é preciso convencer também o \imprimirorientador.}

\hypersetup{
    hidelinks,
    pdftitle={\imprimirtitulo},
    pdfauthor={\imprimirautor},
    pdfsubject={\imprimirpreambulo},
    pdfkeywords={\palavrasChave},
    pdfcreator={LaTeX com abnTeX2 usando Overleaf}
}