\documentclass[12pt, openright, oneside, a4paper, english, brazil]{abntex2}
\usepackage[
    alf,
    abnt-emphasize=bf % títulos em negrito na bibliografia
]{abntex2cite} % Citações padrão ABNT
\usepackage[T1]{fontenc}

\usepackage{fontspec}
\setmainfont{DejaVu Serif}
\setromanfont{DejaVu Serif}
\setsansfont{DejaVu Sans}
\renewcommand{\ABNTEXchapterfontsize}{\LARGE} % default: Huge
\renewcommand{\ABNTEXsectionfontsize}{\Large} % default: Large
\renewcommand{\ABNTEXsubsectionfontsize}{\large} % default: large

\usepackage{tocloft}
\usepackage{indentfirst} % Indenta a primeira linha de um bloco de texto com um ou mais parágrafos. Apenas afeta a primeira linha.
\usepackage{hyperref}
\usepackage{float}

\expandafter\def\expandafter\UrlBreaks\expandafter{\UrlBreaks\do\-\do\.\do\_} % faz quebra de linha das URL em - . e _ para não dar problemas de URL longa estourar o limite de largura de página.

\instituicao{Instituto Superior de Altíssima Reverência e Santidade Aplicada}
\titulo{Monografia teológica reformada}
\newcommand{\subtema}{Modelo estruturado segundo as normas da ABNT e os princípios da Igreja Presbiteriana do Brasil}
\autor{Eclesionildo Reverendíssimo Green dos Calvinuston}
\data{2025}
\local{Vila Predestinada do Norte}
\newcommand{\dataAprovacao}{\rule{1cm}{0.8pt} de \rule{3cm}{0.8pt} de \imprimirdata}
\newcommand{\palavrasChave}{unção acadêmica. homilética coach. conferência do ego. vaidade teológica. culto da estética.}
\newcommand{\keywords}{church, Christian ethics, technology, privacy, data security.}
\orientador{Rev. Agostinélson da Confissão Imutável}
\tipotrabalho{Monografia}
\preambulo{Trabalho acadêmico elaborado para a \textbf{\imprimirinstituicao}, como parte do rito de passagem reformado e obrigatório, sob a gloriosa orientação do inesquecível \imprimirorientador, também conhecido como \textit{``aquele que corrige com amor e a caneta vermelha''}.}
\newcommand{\notaApresentacaoProjeto}{Pré-projeto monográfico apresentado à \textbf{\imprimirinstituicao}, porque, segundo o regulamento, o chamado não basta, é preciso convencer também o \imprimirorientador.}

\hypersetup{
    hidelinks,
    pdftitle={\imprimirtitulo},
    pdfauthor={\imprimirautor},
    pdfsubject={\imprimirpreambulo},
    pdfkeywords={\palavrasChave},
    pdfcreator={LaTeX com abnTeX2 usando Overleaf}
}

\begin{document}

\newpage
\thispagestyle{empty}

\begin{center}
    \textbf{\MakeUppercase{\imprimirinstituicao}}
    \vskip 6cm
    \textbf{\MakeUppercase{\imprimirautor}}
    \vskip 6cm

    \textbf{\MakeUppercase{\imprimirtitulo}:} \subtema

    \vfill
    \imprimirlocal \\
    \imprimirdata
\end{center}
\newpage

\begin{center}
    \textbf{\MakeUppercase{\imprimirautor}}
    \vskip 5.5cm

    \textbf{\MakeUppercase{\imprimirtitulo}:} \subtema

    \vskip 5.5cm
\end{center}
    \begin{flushright}
        \begin{minipage}{0.55\textwidth}
            Projeto monográfico apresentado como requisito parcial para o curso de Bacharelado em Teologia, com o objetivo de expor a proposta de pesquisa a ser desenvolvida na monografia final, incluindo tema, objetivos, fundamentação teórica e metodologia.
        \end{minipage}
    \end{flushright}
    \vskip 3.0cm
\begin{center}
    \vfill
    \imprimirlocal \\
    \imprimirdata
\end{center}
\newpage
\thispagestyle{empty}

\begin{center}
    {\small Este documento está licenciado sob a licença \href{https://creativecommons.org/licenses/by-nc-sa/4.0/}{Creative Commons Attribution-NonCommercial-ShareAlike 4.0 International}.\\
    Você pode copiar, modificar e distribuir esta obra, desde que atribua o crédito apropriado, \textbf{não a utilize para fins comerciais} e distribua as obras derivadas sob a mesma licença.}
\end{center}


\newpage
\pagestyle{plain}
\pagenumbering{arabic}
\renewcommand{\baselinestretch}{1.5}
\normalsize

\noindent
\textbf{Tema:} \imprimirtitulo

\noindent
\textbf{Sinopse:} Este projeto apresenta um modelo orientativo para elaboração de projetos monográficos em seminários teológicos. Ele inclui estrutura sugerida, linguagem acadêmica compatível com a tradição reformada e um toque irônico para manter a sanidade ao longo da jornada. O conteúdo serve como base para seminaristas que desejam organizar suas ideias sem perder a ortodoxia — nem os prazos.

\noindent
\textbf{Problema:} Como estruturar uma pesquisa teológica coerente, com base bíblica, clareza metodológica e risco mínimo de heresia ou reprovação?

\noindent
\textbf{Hipóteses:}
\begin{enumerate}
  \item Um tema claro e biblicamente fundamentado facilita a construção de uma monografia sólida;
  \item A metodologia bem definida é metade da salvação acadêmica;
  \item A fidelidade doutrinária não exclui organização, revisão textual e formatação conforme a ABNT.
\end{enumerate}

\noindent
\textbf{Justificativa:} Muitos seminaristas dominam a Teologia Sistemática, mas tropeçam na Introdução Metodológica. Este modelo busca servir como guia prático e teológico, ajudando o estudante a escrever com reverência e coerência, sem confundir argumentação exegética com devocional pessoal (embora ambos tenham seu lugar).

\noindent
\textbf{Objetivo geral:} Oferecer uma estrutura exemplo para projetos monográficos em seminários teológicos, conciliando ortodoxia, clareza e bom humor.

\noindent
\textbf{Objetivos específicos:}
\begin{itemize}
  \item Demonstrar a aplicação de linguagem acadêmica em contexto teológico;
  \item Ilustrar a organização formal de um projeto segundo a ABNT;
  \item Fornecer um roteiro que o seminarista possa adaptar ao seu próprio tema.
\end{itemize}

\noindent
\textbf{Fundamentação teórica:} A estrutura eclesiástica presbiteriana exige que tudo seja feito “com decência e ordem” (\textit{1Co 14.40}, NAA). Assim também deve ser com os trabalhos acadêmicos. A CFW, em seu capítulo I, destaca que toda a verdade está na Escritura — mas a bibliografia organizada ajuda bastante.

Calvino afirma que “Deus fala com clareza”, e, portanto, o seminarista também deveria tentar \cite[p.~453]{calvinoInstitutas}. A escrita teológica exige responsabilidade, humildade, revisão gramatical e — ocasionalmente — café e jejum.

\noindent
\textbf{Sumário preliminar:}
\begin{itemize}
  \item \textbf{Capítulo 1} – O tema e sua fundamentação bíblico-teológica;
  \item \textbf{Capítulo 2} – Discussão doutrinária ou histórica relevante;
  \item \textbf{Capítulo 3} – Aplicações e implicações pastorais ou práticas.
\end{itemize}

\noindent
\textbf{Metodologia:}
\begin{enumerate}
  \item Levantamento bibliográfico de fontes confiáveis e com ortodoxia comprovada (preferencialmente reformada);
  \item Organização do conteúdo em introdução, desenvolvimento e conclusão;
  \item Aplicação das normas da ABNT, mesmo quando a vontade for ignorá-las;
  \item Submissão à orientação e à correção, como exercício de santificação acadêmica.
\end{enumerate}

\noindent
\textbf{Cronograma:}

\begin{table}[htbp]
    \centering
    \renewcommand{\arraystretch}{1.3}
    \begin{tabularx}{\textwidth}{|c|X|}
        \hline
        \textbf{Data} & \textbf{Etapa do Cronograma} \\
        \hline
        Janeiro  & Etapa de orientação do trabalho para definição de conteúdo e forma \\
        \hline
        6 a 7 de fevereiro  & Apresentação projeto de monografia \\
        \hline
        15 de março  & Entrega do 1º capítulo \\
        \hline
        15 de abril  & Entrega do 2º capítulo \\
        \hline
        15 de maio  & Entrega do 3º capítulo \\
        \hline
        15 de junho  & Entrega das Considerações Finais e Introdução \\
        \hline
        15 de julho  & Entrega completa ao orientador para revisão parcial \\
        \hline
        15 de agosto  & Entrega completa ao orientador para revisão final \\
        \hline
        15 de setembro  & Entrega completa ao orientador \\
        \hline
        30 de setembro  & Entrega ao seminário \\
        \hline
        1 a 15 de outubro  & Defesa em Banca \\
        \hline
        28 de novembro  & Entrega do trabalho após as correções da Banca Examinadora \\
        \hline
    \end{tabularx}
    \caption{Cronograma do trabalho}
    \label{tab:cronograma}
\end{table}
\renewcommand{\bibsection}{%
  \newpage
  \chapter*{\bibname}
  \addcontentsline{toc}{section}{{\MakeUppercase{\bibname}}}
}

% \bibliographystyle{abntex2-alf}
% \bibliographystyle{abntex2-num}
\bibliography{conteudo/referencias/bibliografia}

\end{document}
